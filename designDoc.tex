% Created 2025-09-06 Sat 12:57
% Intended LaTeX compiler: pdflatex
\documentclass[11pt]{article}
\usepackage[utf8]{inputenc}
\usepackage[T1]{fontenc}
\usepackage{graphicx}
\usepackage{longtable}
\usepackage{wrapfig}
\usepackage{rotating}
\usepackage[normalem]{ulem}
\usepackage{amsmath}
\usepackage{amssymb}
\usepackage{capt-of}
\usepackage{hyperref}
\author{Stephen}
\date{\today}
\title{To-Do List Application}
\hypersetup{
 pdfauthor={Stephen},
 pdftitle={},
 pdfkeywords={},
 pdfsubject={},
 pdfcreator={Emacs 30.2 (Org mode 9.7.11)}, 
 pdflang={English}}
\begin{document}
\maketitle

\tableofcontents

\pagebreak
\section{ A Simple Design Document}
\label{sec:org72ed0f9}

\section{Introduction}
\label{sec:org65bcd96}

This document outlines the high-level design for a simple command-line interface (CLI) to-do list application.
The application will serve as a foundational tool for a single user to manage, track, and filter their daily tasks.
The design will follow a modular approach, separating core business logic from the user interface.
\subsection{Core Features}
\label{sec:org29f956b}

The application will support the following key features:

Task Management:
\begin{enumerate}
\item Adding a new task with a description and a due date.
\item Marking an existing task as complete.
\item Deleting a task by its unique identifier.
\end{enumerate}

Task Viewing:
\begin{enumerate}
\item Listing all tasks in a human-readable, formatted string.
\item Filtering tasks to show only those that are completed or incomplete.
\end{enumerate}
\section{Data Model}
\label{sec:orgb19c8a8}

The application's state will be managed by a TodoManager struct, which contains a collection of Task structs.
The data model is as follows:

\subsection{Task}
A Task represents a single to-do item.

\begin{enumerate}
\item ID (int): A unique identifier for the task.
\item Description (string): The description of the task.
\item Completed (bool): A boolean flag indicating whether the task is complete.
\item DueDate (time.Time): The due date of the task.
\end{enumerate}

\subsection{TodoManager}

The TodoManager holds and manages all tasks.

\begin{enumerate}
\item tasks (map[int]*Task): A map that stores pointers to Task structs, with the task ID as the key.
\end{enumerate}
This allows for efficient lookup and modification.

\begin{enumerate}
\item count (int): An integer to generate new, unique task IDs.
\end{enumerate}
\section{Architectural Considerations}
\label{sec:org21f8ae5}
The application will follow a clear separation of concerns:
\subsubsection{Business Logic:}
\label{sec:orgd1d484b}
The TodoManager and Task structs handle all core logic, such as adding, completing, and filtering tasks.
This logic is fully tested via the TDD process.
\subsubsection{User Interface:}
\label{sec:org3c922a5}
A separate main function (not yet implemented) will handle user input from the command line and display output
by calling the business logic methods. This keeps the core logic clean and testable, independent of how the user
interacts with it.
\end{document}
